\documentclass{article}
\usepackage[utf8]{inputenc}
\usepackage[T1]{fontenc}
\usepackage{geometry}
\geometry{a4paper}
\usepackage{helvet}
\renewcommand{\familydefault}{\sfdefault}
\title{A Foundation for Local Multiplayer Gaming Using Smartphones as Wireless Controllers}
\author{Authored by Group 107 (DAY)}
\date{}
\setlength{\topmargin}{-1cm}
\begin{document}
\maketitle
\section{Abstract}
The purpose of this theory is to give an in-depth technological analysis of a multiplayer local video gaming system designed to let players use their smartphones as controllers, rather than the traditional gaming controllers. The system consists of two interconnected parts: a program (video game) running on a personal computer that acts as a game console, and a smartphone application that turns phones into game controllers, enabling the user to interact with the game in new, creative and exciting ways.


\section{Introduction}
In today's modern society most people have access to a smartphone \cite{key:1}. While many use their
smartphones for introverted activities such as surfing the web or messaging distant friends,
surprisingly few applications allow individuals to use their phone to engage with others in the
same physical space.

This situation is problematic for several reasons. Perhaps most noticeably, smartphones can serve
as an excuse not to participate in discussions and other types of face to face interaction. A
worrying example is the interaction and responsiveness between parents and their children,
which might suffer due to excessive smartphone usage \cite{key:2}. We aim to improve upon the situation
by creating a virtual gaming system, where smartphones are used as the controllers and a
computer acts as the console. This will allow smartphones to be used for social activities and
might even strengthen the bonds of parents and children if they play together.


\subsection{Muse and enthusiasm}
There are already products on the market that provide similar ways of controlling video games such as the Nintendo Wii U gamepad, Xbox SmartGlass (a companion application for the Xbox
360 and Xbox One video game consoles. Xbox SmartGlass allows players to control console
applications using a smartphone or tablet).

Unlike the Xbox SmartGlass application, our console makes the phone the primary input device,
and unlike Wii U’s gamepad that only one player in the group can use, our system allows each
player to have their own screen. This opens up for new and more innovative ways for the users to
interact with the games. Furthermore, people carry their smartphones with them at all times,
ensuring there will be enough controllers available to meet the demand.


\subsection{Scope and contributions} 
Smartphones and computers come in many varieties, with different hardware and software
configurations. In order to serve a broad audience it is important to build a general platform that
can accommodate as many devices as possible. The aim of this project is to build a proof of
concept platform. Thus, we only consider the Android operating system on the mobile side and
the Microsoft Windows operating system on the computer. However, in the future we want the
system to be platform independent.

This project has two main contributions. The design and development of the system (Chapter
3) and the implementation of the platform (Chapter 4). Each of these contributions will be addressed in the following chapters.

\section{Design and Development} 
This chapter describes the design of the platform architecture, followed by an brief discussion
of what libraries and languages are used, and why.

\subsection{Designing a platform}
The design decisions that are made when designing the platform are discussed in this section. It
contains the requirements of the system as well as a motivation for using remote code execution
(RCE) and how it is realized on the platform.

There are several hardware components in a smartphone which make it an interesting input
device for games. The touch screen, accelerometer and camera all present different opportunities
in terms of user interaction. Thus, a functional requirement of the system is to be able to read
input from a number of hardware types, including the ones mentioned above and more. The
system also needs a way to present contextual information on the touch screen.

In order to cater for different types of smart phones, with different operating systems, the design
of the system needs to be created with platform-independence in mind. To better accommodate
the needs of the intended audience (attendants of a social event); it is of utter importance that the
process of setting up and starting a game is as quick and simple as possible.
Given the requirements presented above, a basic system architecture can be created. A
fundamental version of the mobile application needs to:
\begin{itemize}
\itemConnect to the game console (in our case, the computer)
\itemTransmit input information
\item Show graphical user interface
\end{itemize}

\subsection{Remote code execution}
In this subsection we will discuss what remote code execution is and how the platform makes use
of it. 

The server is responsible for managing all game and controller logic, as well as the graphical
interfaces for both the server and the smartphone controllers. The parts of the code that concern
the smartphone controllers will (once a connection between the player's phone and the server has
been established) be sent over the connection, received by the phone and executed directly
afterwards.
This process, where code is first sent from one computer to another, and then executed, is
commonly referred to as remote code execution, or RCE in short \cite{key:3}. It is a fairly well-known
method which can and has been used to perform potentially illegal actions in poorly secured IT
systems. There are of course legitimate, non-malicious, applications for RCE as well. A good
example of this is online systems for learning new programming languages. The user in such a
system is typically asked to enter some lines of code in the relevant language, which is then sent
and executed on the server, yielding a result which is then sent back to the user. This makes it
possible to get a hands-on experience with a language, without having to install a compiler,
virtual machine or other additional software on the computer. An example of such a system is
Try Clojure \cite{key:4}.


\section{Platform}
This chapter describes the core of the platform, starting with what is perhaps the most critical
part - the network. It doesn't talk much about the Simple Data Language used for serialization, and
 the game related technologies concerning graphics and sound. 

\subsection{Network}
One of the most, if not the most important part of the platform, is the network implementation.
This section will discuss the various methods and problems associated with implementing a local
client-server model, based on synchronous and asynchronous usage of the transmission control
protocol (TCP).

\subsubsection{Client-server model}
A goal of the project is for the computer to emulate a video game console and the smartphones to
emulate game controllers. The controllers of normal video game consoles do not communicate
with each other. This applies to our system as well, and leads to a classic client-server network
model. The computer (the console) acts as the server and the smartphones (the controllers) act as
clients. Given this there is no direct communication between the phones. Any possible phone to
phone communication has to go via the computer. In the network section, the word client refers
to a smartphone connected to the local network and server refers to the computer in the local
network which the clients connect to.

\subsubsection{Bluetooth versus Wi-Fi}
The computer and the phones need to be connected wirelessly in order for gameplay to be
practical. There are two major competing standards for consumer grade wireless communication,
Wi-Fi and Bluetooth. Both of them have their distinct advantages and disadvantages which we
present below.
\begin{enumerate}
\item  While Bluetooth is available in most laptops, it is not available in all desktop computers.
Even if a desktop computer does not support Wi-Fi, it is probably connected to a router in
the local wireless network via an Ethernet cable, and can thus be connected to. Therefore,
Wi-Fi is more prevalent than Bluetooth.

\item Bluetooth uses different protocols than the ones given by the Internet Standard (the
protocols in use on Wi-Fi). These protocols are less documented and harder to find
tutorials for.

\item Bluetooth handles server discovery through the somewhat cumbersome Bluetooth pairing
process. Such a process does not exist on Wi-Fi and has to be implemented by the
programmer herself.

\item Bluetooth consumes less power \cite{key:5}.
\end{enumerate}
After evaluating the properties of the two technologies, we came to the conclusion that Wi-Fi
was a better fit for our purposes.

\section{Conclusion}
The number of people who own smartphones is growing at a fast pace, and has been projected to
reach 5.6 billion in 2019 \cite{key:1}. Smartphones include many advanced hardware capabilities that can
be utilized in a large number of ways. This project aims to make use of these devices to the fullest extent possible, by creating a virtual game console which uses smartphones as controllers.

There are also drawbacks with this approach. This is mainly due to the fact that smartphones lack
many important hardware components present in modern day controllers. These include things
like thumb sticks and hardware buttons. More avid players have come to expect these on a game
controller making the transition awkward. However, there are many players who only play
games on their smartphone and these players will have no problem using them on this platform.

\begin{thebibliography}{9}

\bibitem{key:1}
Ericsson. (2013, November) Ericsson Mobility Report. [Online].
http://www.ericsson.com/res/docs/2013/ericsson-mobility-report-november-2013.pdf
\bibitem{key:2}
Caroline J. Kistin, Barry Zuckerman, Katie Nitzberg, Jamie Gross, Margot Kaplan-Sanoff,
Marilyn Augustyn and Michael Silverstein Jenny S. Radesky, "Patterns of Mobile Device
Use by Caregivers and Children During Meals in Fast Food Restaurants," Pediatrics, Mar.
2014.
\bibitem{key:3}
 Hannes Holm, Mathias Ekstedt Theodor Sommestad, "Estimates of success rates of remote
arbitrary code execution attacks," Information Management & Computer Security, vol. 20,
no. 2, pp. 107-122, 2012.
\bibitem{key:4}
 Anthony Grimes et al. (2014, May) tryclojure. [Online]. https://tryclj.com/
\bibitem{key:5}
Yu-Wei Su, Chung-Chou Shen Jin-Shyan Lee, "A Comparative Study of Wireless
Protocols: Bluetooth, UWB, ZigBee, and Wi-Fi," in Industrial Electronics Society, 2007.
IECON 2007. 33rd Annual Conference of the IEEE, Taipei, 2007, pp. 46 - 51.

\end{document}
